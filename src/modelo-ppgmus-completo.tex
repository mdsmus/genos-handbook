\documentclass{ppgmus}

\titulo{Minha Dissertação}
\autor{José das Couves}
\ano{2008}
\tipo{mestrado}
\area{Composição}
\lugar{Salvador}
\orientador{Prof. Dr. Foo Bar}
\mes{Janeiro}

\textoDedicatoria{A fulano, ciclano e beltrano.}

\textoEpigrafe{Bla bla bla bla bla bla bla bla bla bla bla bla bla bla
  bla bla bla bla bla bla bla bla bla bla bla bla bla bla bla bla bla
  bla bla bla bla bla bla bla bla bla bla bla bla bla bla bla bla bla
  bla bla bla.}

\autorEpigrafe{Confucio}

\dataDefesa{Salvador}
\localDefesa{19 de dezembro de 2004}

\bancaiNome{Jamary Olveira}
\bancaiTitulo{Doutor em Composição}
\bancaiEstudo{\textit{University of Texas at Austin}, EUA}
\bancaiAtuacao{Universidade Federal da Bahia}

\bancaiiNome{Lucas Robatto}
\bancaiiTitulo{Doutor em Flauta}
\bancaiiEstudo{Universidade de Washington, EUA}
\bancaiiAtuacao{Universidade Federal da Bahia}

\bancaiiiNome{Wellington Gomes da Silva}
\bancaiiiTitulo{Doutor em Música}
\bancaiiiEstudo{Universidade Federal da Bahia (UFBA)}
\bancaiiiAtuacao{Universidade Federal da Bahia}

\begin{document}

\maketitle
\tableofcontents

\chapter*{Agradecimentos}
\label{cha:agradecimentos}

\chapter*{Glossário}
\label{cha:glossario}

\chapter*{Resumo}
\label{cha:resumo}

\chapter*{Abstract}
\label{cha:abstract}

%% é importante ter esse mainmatter para indicar que começa o corpo do
%% texto
\mainmatter

\chapter{Introdução}

\section{teste1}

\newpage

\section{teste2}

\newpage

\section{teste3}

\appendix

\chapter{Dados estatísticos}
\label{cha:dados-estatisticos}

%%%% Bibliografia

\addcontentsline{toc}{chapter}{\bibname}
\bibliographystyle{plain}
\bibliography{composition}

\end{document}
